\usepackage{listings}

\usepackage{xcolor}
% \newcommand{\meta}[1]{{\color{blue}\textbf{[}#1\textbf{]}}}

\usepackage{tikz}
\usetikzlibrary{arrows}

\usepackage{mathtools}
\DeclarePairedDelimiter\ceil{\lceil}{\rceil}
\DeclarePairedDelimiter\floor{\lfloor}{\rfloor}

\usepackage{prettyref}
\newcommand{\pref}{\prettyref}
\newrefformat{thm}{Theorem~\ref{#1}}
\newrefformat{cor}{Corollary~\ref{#1}}
\newrefformat{lem}{Lemma~\ref{#1}}
\newrefformat{cha}{Chapter~\ref{#1}}
\newrefformat{sec}{Section~\ref{#1}}
\newrefformat{app}{Appendix~\ref{#1}}
\newrefformat{tab}{Table~\ref{#1}}
\newrefformat{fig}{Figure~\ref{#1}}
\newrefformat{alg}{Algorithm~\ref{#1}}
\newrefformat{exa}{Example~\ref{#1}}
\newrefformat{def}{Definition~\ref{#1}}
\newrefformat{li}{Line~\ref{#1}}
\newrefformat{eq}{Equation~\ref{#1}}
\newrefformat{exa}{Example~\ref{#1}}

% Language semantics layout adapted from
% https://dl.acm.org/doi/pdf/10.1145/3371106

\newcommand{\leif}[3]{\left(#1\right)\hspace{1mm}?\hspace{1mm}\left(#2\right)\hspace{1mm}:\hspace{1mm}\left(#3\right)}
\newcommand{\whilevarnothalt}[1]{\mathbf{while}\hspace{1mm}\mathtt{#1}\neq\mathbf{halt}\lkw{do}}
\newcommand{\lkw}[1]{\mathbf{#1}\hspace{1mm}}
\newcommand{\lfn}[1]{\mathbf{#1}}
\newcommand{\lvar}[1]{\mathtt{#1}}
\newcommand{\lval}[1]{\mathbf{val}(#1)}
\newcommand{\lnext}[2]{\mathbf{next}(#1, #2)}
\newcommand{\lhalt}{\mathbf{halt}}
\newcommand{\lnull}{\mathbf{NULL}}
\newcommand{\lisnull}[1]{\mathbf{isnull}(#1)}
\newcommand{\qqquad}{\qquad\quad}

\usepackage{mathtools}

\DeclarePairedDelimiter\abs{\lvert}{\rvert}%
\DeclarePairedDelimiter\norm{\lVert}{\rVert}%

\usepackage[linesnumbered]{algorithm2e}
\SetKwFor{While}{while (}{) $\lbrace$}{$\rbrace$}
\SetKwIF{If}{ElseIf}{Else}{if (}{) $\lbrace$}{$\rbrace$ else if}{$\rbrace$ else $\lbrace$}{$\rbrace$}

% https://stackoverflow.com/questions/4439605/c-source-code-in-latex-document
\lstset{
  language=C,                     % choose the language of the code
  numbers=left,                   % where to put the line-numbers
  stepnumber=1,                   % the step between two line-numbers.        
  numbersep=2pt,                  % how far the line-numbers are from the code
  backgroundcolor=\color{white},  % choose the background color. You must add \usepackage{color}
  showspaces=false,               % show spaces adding particular underscores
  showstringspaces=false,         % underline spaces within strings
  showtabs=false,                 % show tabs within strings adding particular underscores
  tabsize=2,                      % sets default tabsize to 2 spaces
  captionpos=b,                   % sets the caption-position to bottom
  breaklines=true,                % sets automatic line breaking
  breakatwhitespace=true,         % sets if automatic breaks should only happen at whitespace
  % title=\lstname,                 % show the filename of files included with \lstinputlisting;
  basicstyle=\scriptsize,
}

\usetikzlibrary{calc,shapes.multipart,chains,arrows}

\newcommand\tzconnect[3]{
    \draw (#1.south west) -- (#2.north west);
    \draw (#1.south east) -- (#3.north east);
    \draw (#2.south west) -- (#3.south east);
}
\newcommand\tznext[2]{
    \draw[*->] let \p1 = (#1.two), \p2 = (#1.center) in (\x1,\y2) -- (#2);
}
\newcommand\tzstripe[3]{
    \draw[ultra thick,#3] ([yshift=#2]#1.north west) -- ([yshift=#2]#1.north east);
}
\newcommand\tzstriped[3]{
    \draw[ultra thick,#3,-|] ([yshift=#2]#1.north west) -- ([yshift=#2]#1.north east);
}
\newcommand\tzstripel[4]{
    \draw[ultra thick,#4] ([yshift=#3]#1.north west) -- ([yshift=#3]#2.north east);
}
\newcommand\tzstripedl[4]{
    \draw[ultra thick,#4,-|] ([yshift=#3]#1.north west) -- ([yshift=#3]#2.north east);
}
\newcommand\tzstripes[3]{
    \draw[ultra thick,#3] ([yshift=#2]#1.south west) -- ([yshift=#2]#1.south east);
}
\newcommand\tzstripesl[4]{
    \draw[ultra thick,#4] ([yshift=#3]#1.south west) -- ([yshift=#3]#2.south east);
}
\usetikzlibrary{decorations.pathreplacing,calligraphy,arrows.meta}

\newcommand\prename{\texttt{pre}}
\newcommand\postname{\texttt{post}}
\newcommand\modname{\texttt{mod}}
